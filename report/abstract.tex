
%=====================================================================
\chapter{Abstract} \label{chp:abstract}
%=====================================================================

The increasing size and dimensionality of datasets in the humanities pose new challenges to scholars working with them, including how to establish an overview over the dataset, how to establish new hypotheses and how to test them. Material, pattern, and texture usage in moving image is an attractive example of such multi-dimensional datasets in film studies, as an almost infinite number of combinations thereof are possible.\\ 
This thesis proposes interactive data visualization and exploration of film features as a solution to assist film researchers in the investigation of relationships between material, pattern, textures, and a wide range of qualitative analysis labels in a large film database. Using an array of visual components in combination with exhaustive computational methods and an interactive user-interface, the proposed solutions show to be effective in conveying the global and local structure of the dataset as well as the relationship between different feature groups of features. Finally, the resulting visualization tool is embedded into the well-established research framework of the FilmColors project.

