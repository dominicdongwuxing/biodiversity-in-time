
%=====================================================================
\chapter{Abstract} \label{chp:abstract}
%=====================================================================

The explosion of data in recent years has inspired interesting visualizations to communicate data-driven ideas in intuitive and memorable ways. Current project addresses the topic of biodiversity through time. With the collective effort of paleobiologists, online databases have been constructed to store locations and taxonomy identifications of fossil records collected all over the world. In the meantime, recent progress in Earth science has tracked how continents continuously shifted in the past 410 million years in high spatial-temporal resolution. Therefore, it is now possible to design a visualization that combines both data to illustrate how life and Earth together changed in the course of time. \\

This thesis purposes a complete workflow from data processing to data visualization in a web-application. We clean all 1.3 million animal and plant fossil data in the PaleoBiology Database, and organize their taxonomic information into a tree of life.  We reconstruct fossil locations at various ancient time points using a published simulation model on tectonic movement. The final result provides a rich and engaging visualization prototype that allows users to explore the tree of life, Earth geography, and fossil distribution in the world during Phanerozoic Eon. To our knowledge, this is the first interactive visualization that combines tectonic motion and tree of life across time, thereby providing users a holistic view of how different forms of life developed and where they lived in various periods of Earth history.
