
%=====================================================================
\chapter{Zusammenfassung} \label{chp:abstract_german}
%=====================================================================

Computergestützte Methoden sind ein integraler Bestandteil in der heutigen Datenbeschaffung der Humanities geworden. Die Dimensionalität und Grösse der Datensätze, welche solche Methoden generieren, stellen Wissenschaftler jedoch vor Herausforderungen, insbesondere wie ein Überblick über den Datensatz geschaffen werden kann und auf welchen Wegen neue Hypothesen aufgestellt, sowie überprüft werden können. Die Verwendung von Materialien, Texturen und Muster im Film ist dabei ein perfektes Beispiel für solchen, multidimensionalen Datensatz. \\
Diese Thesis zeigt auf, wie interaktive, auf Datenerkundung ausgerichtete Visualisierungstechniken dabei helfen können, Beziehungen zwischen der Verwendung von Materialien, Texturen, Mustern und einer grossen Fülle an qualitativer Klassifizierung zu identifizieren. 
Dabei wird eine Reihe an Visualisierungskomponenten kombiniert mit einer rekursiv erstellten t-SNEs dazu verwendet, nicht nur globalen und lokalen Strukturen eines Datensatzes aber auch deren Beziehung zu anderen Eigenschaften und Attributen dargestellt. Die daraus resultierende Visualisierungsapplikation ist darüber hinaus in das Forschungsprojekt FilmColors eingebettet und bindet deren Visualisierungen nahtlos ein.



